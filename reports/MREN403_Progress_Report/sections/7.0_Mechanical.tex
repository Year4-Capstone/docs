\subsection{Mechanical Adaptations Subsystem Progress}
\label{sec:mechanical}

Foundational progress has been made in the Mechanical Adaptations Subsystem, as highlighted in \autoref{tab:table_ma}, and the updated Gantt chart is illustrated in \autoref{fig:gantt_ma}. 
\begin{table}[H]
\centering
\caption{Mechanical Adaptations (MA) Progress Summary.}
\begin{tabular}{p{3cm} p{5cm} c c c}
\toprule
ID & Name & Progress & Task Mod. & Sched. Mod. \\
\midrule
FLT-MA-001 & Hardware Acquisition & COMPLETED & No & No \\
AMR-MA-001 & Payload Security & Yes & Yes & EXTENDED \\
AMR-MA-002 & Mounted Sensors & Yes & No & DELAYED \\
FLT-MA-002 & RC Robots & Yes & No & No \\
\bottomrule
\end{tabular}
\label{tab:table_ma}
\end{table}

\begin{figure}[H]
    \centering
    \includegraphics[width=1\linewidth]{reports/MREN403_Progress_Report/images/Mechanical Adaptations Gantt.png}
    \caption{Mechanical Adaptations (MA) Updated Gantt Chart.}
    \label{fig:gantt_ma}
\end{figure}

The team faced a challenge of mounting all the required sensors and payload on very limited mounting points. To address this, the team designed a solid top that provides further space for mounting sensors (\textbf{AMR-MA-002}), computing units, and any required accessories. Additionally, the solid top provides a sturdy surface for easily mounting and securing the payload (\textbf{AMR-MA-001}). Another challenge arose when determining the optimal location for sensor placement. To mitigate this, the team opted to use plywood for the solid top, allowing them to test different sensor setups incredibly easily. The team has one of the two RC cars functional (\textbf{FLT-MA-002}) and has acquired all the hardware needed to date (\textbf{FLT-MA-001}). 


\subsubsection{Schedule Modifications}

\begin{itemize}
    \item \textbf{AMR-MA-001:} Extended to accommodate the introduction of the solid top.
    \item \textbf{AMR-MA-002:} Delayed due to uncontrollable delay in \textbf{FLT-MA-001}.
\end{itemize}

