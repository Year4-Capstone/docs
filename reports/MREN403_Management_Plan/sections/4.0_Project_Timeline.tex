\section{Project Timeline}
\label{sec:project_timeline}

The team utilizes Jira to plan and monitor the progress of the project effectively. 
Jira automatically generates a Gantt chart (\autoref{fig:timeline}), which showcases the project’s timeline and features. 
Due to uncertainties like lead times for parts and feature integration, and the need for collaboration between multiple team members, the Gantt chart deviates from the typical format and includes more features completed simultaneously. 
This design was chosen to address these and future uncertainties, allowing the team to start working on a feature even before its prerequisite is complete.

Additionally, Jira facilitates task tracking and breaking the features down into smaller, more manageable goals. 
This ensures that each deliverable is practical and measurable. 
Tasks are created when a deliverable is expected to take four or more hours of work or when collaboration between two or more team members is needed for its completion. 
The Gantt chart provides a high-level overview of the project’s goals and their features, while the individual tasks allow the team to track detailed progress and remain on schedule.

\subsection{Tasks Breakdown}
\label{sec:tasks_breakdown}
A complete breakdown of tasks associated with each functional requirement is shown in \autoref{tab:tasks}.
Each task is tied to a specific feature ID, ensuring traceability between implementation progress and the high-level functional requirements described in \autoref{sec:functional_requirements}.