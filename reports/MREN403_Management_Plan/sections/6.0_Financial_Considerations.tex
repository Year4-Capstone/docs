\section{Financial Considerations}
\label{sec:financial considerations}

\subsection{Capital Costs and Proposed Budget}
\label{sec:capital_costs}
The total estimated prototype cost, excluding components provided with the Ibex platform~\cite{caldwell2018design} or already owned by the team, is approximately \$260 CAD. The breakdown of the prototype cost is summarized in --%\autoref{tab:prototype_cost}
. 
Prices were obtained from Digi-Key and Mouser. This amount represents the project’s capital cost, which includes purchased components such as two 12 V 12 Ah SLA batteries and other planned hardware purchases required for sensing, communication, and integration. The distribution of the supplied budget, illustrating the split between spent, planned purchases, and remaining budget, is shown in --%autoref{fig:budget_distribution}
.
A detailed breakdown of the budget allocated for purchasing by subsystem and shipping can be found in --%\autoref{fig:subsystem_budget}
.

\subsection{Prototype Development and Design Iteration, and Validation}
\label{sec:prototype_development}
The prototype is built on the Ibex mobile platform, integrating new sensing, communication, and software stack while leveraging existing hardware to reduce cost and complexity. The payload subsystem will serve as a visual representation of the intended real-world payload, emphasizing the real-world constraints on the design. The Payload will be 3D-printed with filament and printing resources provided by the university, minimizing additional material costs. Subsystem iteration and validation will be initially performed in ROS 2 and Gazebo simulation environments, where possible, before hardware testing. This simulation-focused approach reduces physical wear and lowers the likelihood of component failure, thereby minimizing replacement expenses.

\subsection{Operating, Societal, and Life-Cycle Costs}
\label{sec:operating_costs}
Operating costs for the project are minimal, limited primarily to charging the dual 12 V battery system and routine mechanical maintenance. Since the payload system is a visual representation of the intended design, there will be no operational cost associated with it within the current scope of the project. The system aims to reduce human exposure to hazardous environments and accelerate rescue operations, providing net societal and safety benefits relative to its small environmental footprint. From a life-cycle cost perspective, the system’s costs are split into acquisition, operation, maintenance, and disposal and are summarized in --%\autoref{tab:lifecycle_costs}
. 
These stages capture long-term costs, including preventive servicing and sustainable end-of-life practices.