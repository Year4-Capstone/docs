\section{Resource Management}
\label{sec:resource_management}

Effective resource management ensures that the team’s digital infrastructure supports collaboration, version control, and accountability across all subsystems. Development resources are organized to align with the sprint-based framework and integrated risk controls described in the \nameref{sec:team organization} and \nameref{sec:risk_mitigation} Sections.

A GitHub organization stores all the repositories for the Search \& Support Fleet Robotics project. These repositories manage all source code, hardware design files, and formal documentation. The main repository, \textbf{search-support-robots}, serves as a central repo connected to the following submodules:

\begin{itemize}
    \item \textbf{dev-tools}: environment setup scripts and instructions.
    \item \textbf{simulation-environment}: Gazebo worlds and URDF configurations.
    \item \textbf{ros-packages}: ROS 2 nodes, launch, and configuration files.
    \item \textbf{mesh-network}: communications and telemetry.
    \item \textbf{hardware-design}: CAD models, wiring diagrams, and schematics.
    \item \textbf{docs}: LaTeX source for reports.
\end{itemize}

This modular structure supports the development of concurrent subsystems and bug traceability, while maintaining version control. Each repository shares a standardized branching model (\textbf{main}, \textbf{feature}, \textbf{bugfix}). Branch protection rules prevent direct pushes to \textbf{main} and require peer-approved pull requests to be merged. In addition, GitHub Actions prevent pull requests from being merged into the \textbf{main} branch unless all related requirements pass. The current requirements include building ROS2 packages, compiling Arduino firmware, and generating Overleaf PDFs without error. These version control safeguards serve as a risk mitigation consistent with the \nameref{sec:risk_mitigation} Section, reducing data loss or corrupted code.

Integration between Jira and GitHub provides traceability between pull requests and task IDs derived in the \nameref{sec:functional_requirements} Section. Microsoft Teams serves as the main platform for non-code communication, as discussed in the \nameref{sec:com_chans} Section. Also, Teams stores shared files, meeting notes, and presentation slides. The GitHub organization and all repositories within it are co-owned by the team members to ensure equal access.