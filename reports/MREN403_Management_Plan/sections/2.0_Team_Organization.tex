\section{Team Organization}
\label{sec:team organization}

\subsection{Roles and Responsibilities}
The Search \& Support Fleet Robotics project involves the collaboration of four mechatronics engineering students, with each member contributing unique technical expertise to the development of the robotic fleet and accompanying mesh network. Roles are designed to align with individual strengths while maintaining an equal distribution of workload and opportunities for skill development, see \autoref{tab:member_roles}. Note, these roles are primarily for members’ design focuses; all members will have tasks that cover all subsystems defined in the \nameref{sec:functional_requirements} Section.

\subsection{Project Management Framework}
The team will adopt a lightweight Agile-based framework with two-week sprints to support the design and validation process. This framework emphasizes iterative progress, frequent communication, and rapid adaptation to technical challenges, especially under time, budget, and subsystem constraints.

Each sprint will last two weeks, giving the team ample time to develop key functional requirements and iteratively improve requirements in subsequent sprints. Each sprint will conclude with a meeting with the technical supervisor, Dr. Marshall, where the team will receive feedback to guide their work for the next sprint.
\begin{itemize}
  \item \textbf{Planning} – define \textit{what needs to be done}, \textit{why it matters}, and \textit{how the team will approach it} during the upcoming sprint.
  \item \textbf{Design} – translate sprint goals into technical blueprints, such as schematics, simulations, and system architecture, which guide implementation and determine feasibility.
  \item \textbf{Development} – build or implement the design. Each team member will be given a task to complete during the sprint.
  \item \textbf{Testing} – verify and validate that the product increment meets design requirements, functional goals, and safety standards.
  \item \textbf{Deployment} – finalize the feature highlighted in the sprint goal and integrate it with the rest of the larger system.
  \item \textbf{Review} – reflect on the sprint or project cycle, identifying areas of improvement and success. Feedback from Dr.~Marshall will guide the next sprint.
\end{itemize}
To facilitate a smooth sprint process, the team will utilize multiple tools. A Jira planner will be used to visualize sprint tracking. This planner will be connected to a Gantt chart, which will be used for long-term scheduling and milestone visualization. Lastly, a GitHub repository will be used to maintain version control for code, documentation and design files. 
\subsection{Communication Protocols}
Reliable communication will be central to the success of the Search \& Support Fleet Robots project, as the team will need to develop subsystems concurrently. The team has established clear protocols for meetings and collaboration.  
\subsubsection{Meeting Schedule}
The team will hold weekly Monday meetings focused on project management and coordination. During this session, members will review progress made over the past week and rebalance workloads to ensure even task distribution. The Gantt chart will be updated to reflect current progress and changes to deadlines.
The team will also meet bi-weekly on Tuesdays with the technical supervisor. This meeting will provide opportunities to present current project updates, discuss design reviews, and receive technical guidance. The feedback received will be used to refine the project’s direction, ensuring the team can meet technical and academic expectations. 
Lastly, the team will meet during the pre-allotted weekly workshop time. These sessions will focus on deliverable-oriented work, subsystem creation, and overall system testing. Working during these sessions will ensure the team can meet its weekly targets. Typical activities will include mesh-network testing, motor/sensor calibration and working on project management deliverables.
Wednesdays have been chosen as a flexible overflow day, reserved for additional collaboration and troubleshooting to address any outstanding challenges or prepare for upcoming milestones. 
\subsubsection{Communication Channels}
\label{sec:com_chans}
The team will use Microsoft Teams as the primary communication platform. For immediate coordination, such as hardware testing updates or unexpected schedule changes, iMessage will be used as an urgent communications platform. 
\subsection{Workload Management} 
Due to the complexity of the project, proper workload management and distribution are required to ensure team success. To manage this, larger tasks and features will be broken down into their core components and assigned to subsystem leads. There will be intentional overlap to promote cross-training and ensure knowledge sharing between team members. This will ensure redundancy in knowledge, team adaptability and boost overall project resilience. Task management will be coordinated using a shared Jira planner board, which visualizes the team’s progress and highlights any outstanding work. Workload assignments will be reviewed weekly during the Monday team meetings, and work will be redistributed as needed in case of shifting priorities or emerging challenges. Each subsystem lead will also maintain a deliverable log, which will be reviewed at the start of each sprint to ensure accountability and traceability. This monitored approach will help the team reduce bottlenecks associated with unbalanced workloads and responsibilities. 
\subsection{Conflict Prevention and Resolution} 
The team contract highlights key team behaviors and is as follows. All team members agree to:
\begin{enumerate}[label=\textbf{\Alph*.}]
  \item \textbf{Professional Conduct}
  \begin{itemize}
    \item Communicate respectfully and constructively at all times.
    \item Listen actively and consider multiple viewpoints before making decisions.
    \item Avoid blame – focus on solutions and continuous improvement.
  \end{itemize}

  \item \textbf{Responsibility and Accountability}
  \begin{itemize}
    \item Complete assigned tasks on time and to agreed quality standards.
    \item Maintain transparency in progress by communicating with other team members and updating task logs.
    \item Notify team members of any obstacles that may affect timelines.
  \end{itemize}

  \item \textbf{Collaboration and Participation}
  \begin{itemize}
    \item Attend all scheduled meetings unless excused in advance.
    \item Contribute to work sessions and discussions.
    \item Provide constructive feedback on others’ work in a professional manner.
  \end{itemize}
\end{enumerate}
The team will follow a guideline for conflict resolution. In the event of a conflict, disputes are to be first addressed collaboratively during management meetings. If the conflict persists, the team will take a vote to break the tie. If the issue is still unresolved, the technical supervisor will act as the impartial tie-breaking authority. Conflict prevention will also be supported by transparent workload visibility, peer accountability, and adherence to professional communication standards.

\subsection{Progress Tracking and Reporting }
All project milestones are directly linked to the team’s Gantt chart and course deliverables. The Gantt chart will be updated as required, based on changes made to milestone deadlines. Each subsystem lead will be tasked with updating and monitoring their segment of the chart. Key deliverables monitored through this process include design and analysis documentation, prototyping, subsystem test results and summaries of each meeting. This will ensure all work is accounted for and aligned with project objectives. 
