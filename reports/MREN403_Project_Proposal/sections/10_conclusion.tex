\section{Conclusion}
\label{sec:conclusion}

Mine collapses are one of the most serious risks faced by mining workers, as they often lead to the entrapment of workers and the loss of communication with rescue teams. The Search \& Support Fleet Robotics project addresses these issues by deploying a fleet of autonomous and remotely controlled robots that are connected via a mesh network. These robots are capable of restoring underground communication, detecting and locating trapped workers, monitoring hazardous environments, and delivering support payloads.

The benefit of this solution is that it reduces the need for human rescuers to enter dangerous or unknown environments immediately. \gls{fleet} improves safety for the response team while also increasing the survivability of trapped miners. The mesh network offers the option to integrate additional robots into the system, allowing the system to adapt to future needs.

This project integrates robotics, sensing, and networking into a unified system that advances disaster relief operations in the mining industry. The outcome of this project is both a functional prototype and a framework for autonomous rescue fleets, providing a safer, faster, and more effective emergency response.
