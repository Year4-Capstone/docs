\section{Project Scope}
\label{sec:project_scope}

The Search \& Support Fleet Robotics project addresses the complex engineering challenge of designing a mobile robotic system capable of operating in unpredictable, hazardous post-disaster mining environments, where rapid action is critical for effective relief. The robotic fleet will be comprised of one autonomous robot and two remotely operated support robots to enhance worker safety and response effectiveness during mining emergencies. \gls{fleet} will also rely on a mesh network for effective communication between the robots, thereby overcoming the difficulties presented by the partially obstructed nature of underground mines. Through this mesh network, the robots will be able to relay their respective location and position data, ensuring coordinated navigation, real-time situational awareness, and uninterrupted communication. The primary engineering challenges include designing a robust chassis and drivetrain to navigate uneven and unstable terrain, integrating environmental sensors to detect temperature, humidity, and hazardous gases, implementing human detection systems, such as infrared sensors to read heat signatures, and enabling reliable delivery of life-support \gls{payload}, including oxygen, water, and medical supplies. Power consumption will also need to be monitored to ensure sufficient operational duration, given the estimated mass, motor load, and sensor consumption. The project scope also includes predefining a 3D map of the mine environment, enabling \gls{fleet} to identify \gls{anomalies} within the terrain, detect hazards, and optimize navigation and rescue operations. Payload requirements for the survival needs of up to three \gls{persons} will also need to be investigated. The suspension, drivetrain, and sensors will be selected through a rigorous, systematic, data-driven process to ensure optimal parts are chosen. The team will define the criteria and constraints of each part, gather data, and use quantitative methods, such as weighted decision matrices, to select components. Required resources include robotic hardware (autonomous and \gls{rc} platforms), sensors for environmental monitoring and human detection, batteries and energy management systems, communication routers for mesh networking, and computational resources for navigation, path planning, and fleet coordination. This project presents a comprehensive solution to enhance worker safety and response effectiveness during mining emergencies, while addressing the multidisciplinary engineering challenges inherent in autonomous rescue robotics.
