\section{Assumptions and Constraints}
\label{sec:assumptions_constraints}

Several assumptions and constraints will influence the final design of the Search \& Support Fleet Robotics project. The \gls{amr} will be based on the Offroad Robotics \gls{ibex}.

The project will proceed under the following assumptions. The \gls{ibex} is in a reparable state, allowing for modifications and integration into the system. \gls{fleet} will operate in mining environments where a predefined three-dimensional map of the mine is provided. The \gls{payload} estimate for life support supplies is based on preliminary research. 

The following factors also constrain the project. Since the \gls{amr} is based on the \gls{ibex}, the drivetrain will not be a primary design focus. The \nameref{sec:mvp} is limited to one autonomous robot and two remotely operated robots adapted from \gls{rc} platforms. Regardless of the platform, the drivetrain and chassis must meet the requirements for traversing uneven terrain, slopes, and obstacles. The three robots must comply with safety standards and codes, as outlined in \nameref{sec:codes_regulations}. Environmental sensors will be baseline tested only due to a lack of specialized equipment. 

The project will be constrained by the provided budget and the limited allocatable time due to competing tasks and responsibilities. The forthcoming Management Plan will highlight the resolution to such time and budget related constraints.
