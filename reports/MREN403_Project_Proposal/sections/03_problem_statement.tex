\section{Problem Statement}
\label{sec:problem_statement}

In recent years, the mining industry has experienced significant growth, driven by increasing demand for raw materials used in construction, manufacturing, and clean energy technologies. However, many mining operations face significant challenges related to worker safety. Mining collapses continue to be a serious hazard that workers risk encountering in their daily operations. Collapses may crush, entrap, or suffocate workers, posing severe risks of injury or death, making rapid and effective rescue operations critically important. According to \gls{msha} statistics, between 2000 and 2021, there were 8,800 injuries to underground workers, including 122 fatal events~\cite{cdc_groundfalls}. As global demand for mining continues to rise, the risk of mining-related disasters is expected to increase accordingly, creating an urgent need for advanced search and rescue solutions to address potential collapses. 

Therefore, there is a need to develop a robust fleet of autonomous, mobile search and support robots to aid workers during the event of a collapse. A successful solution would implement a mesh network to facilitate seamless communication among the robotic fleet. These robots could then provide critical support by delivering oxygen tanks, water, and medical supplies to \gls{persons}.
